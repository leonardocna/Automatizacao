\hypertarget{ApB}{
\section{Apêndice A: Respostas aos Quesitos da Requerente}}

\textbf{1.	Queira o I. Perito Judicial transcrever na integra os itens destacados no “Relato da Infração do documento 1 (fls.31/34), AIIM nº 4.037.656, destacando o montante das operações que infringiram o RICMS, e serviram como base para multa punitiva;}

\textbf{Resposta:} Visando prover celeridade e economicidade processual, respeitosamente, a \textbf{Perícia} pede reporte ao AIIM n° \textbf{4.037.656-4} acostado na íntegra às fls. 180/183 dos autos, para visualização do quanto perquirido no atual quesito.

\textbf{2.	Queira o I. Perito Judicial informar, conforme os documentos acostados nos autos, se Agente Fiscal atuante observou integralmente e relatou, os registros contábeis da Autora para fundamentar a lavratura do mencionado Auto de infração e imposição de Multa que originou a Certidão de Dívida Ativa ora executada.}

\textbf{Resposta:} Esclarece este signatário que, conforma explanado no item \textbf{“1.2 Deferimento, objeto e objetivo da Perícia”} do Laudo Pericial Contábil, restringindo-se ao juízo técnico pericial demandando na presente Ação, a \textbf{Perícia} limita o escopo da prova para tão somente confirmação dos valores a título de correção e juros incididos sobre o principal do imposto e valores básicos de multa punitiva pela aplicação da taxa de \emph{14,78\%} adotada na data da lavratura do AIIM em detrimento da Taxa SELIC, não adentrando, portanto, no mérito da questão quando a análise da origem do principal e sua composição, motivo pelo qual, prejudicada a resposta.

\textbf{3.	Queira o I. Perito Judicial informar, qual a penalidade por creditar-se indevidamente do ICMS, com fundamento no item 3 do §1º do artigo 59 do RICMS/2000. INFRINGÊNCIA: Arts. 59, §1°, item 3, arts. 61, art. 64, Inc. I, do RICMS (Dec.45.490/00)}

\textbf{Resposta:} Visando prover celeridade e economicidade processual, respeitosamente, a Perícia pede reporte a resposta ao quesito imediatamente anterior.

\textbf{4.	Queira o I. Perito Judicial demonstrar, se houve observação no documento 1 (fls.31/34), AIIM nº 4.037.656, e quais os termos e condições do Artigo 95, incisos I e II e §§ 1º e 8º, da Lei 6.374/89, na redação dada pela Lei 13.918/09, de 22/12/2009, para a multa e, nos termos do artigo 100 do Decreto nº 54.486/2009, sobre prazo para a defesa;}

\textbf{Resposta:} Atendendo ao quanto perquirido, a \textbf{Perícia} confirma que as referências de legislação sugerida no quesito estão contempladas no AIIM \textit{sub judice} (fls. 31/34 dos autos), de modo que, reproduz a capitulação legal no que concerne as penalidades aplicadas, uma vez que está contemplado no objeto da demanda, vejamos:

\textbf{Artigo 95 Inciso I e II e § 1° e 8° da Lei 6.374/89, redação dada pela Lei 13.918/09}

“(...)

Artigo 95 - Pode o autuado pagar a multa aplicada nos termos do artigo 85 desta lei, com desconto de: (NR)

I - 70\% (setenta por cento), dentro do prazo de 15 (quinze) dias contados da notificação da lavratura do auto de infração; (NR)

II - 60\% (sessenta por cento), dentro do prazo de 30 (trinta) dias contados da notificação da lavratura do auto de infração; (NR)

(...)

§ 1º - Condiciona-se o benefício ao integral pagamento do débito. (NR)

(...)

§ 8º – Tratando-se de penalidade aplicada sobre o valor do imposto, a aplicação dos descontos previstos neste artigo não poderá resultar em penalidade inferior a 25\% (vinte e cinco por cento) do valor do imposto. (NR)

(...)

Artigo 100 do Decreto n° 54.486/2009

(...) 

\textbf{Artigo 100} - Lavrado o auto de infração, terão início os procedimentos de cobrança administrativa, devendo o autuado ser notificado a recolher o débito fiscal, com o desconto de lei, quando houver, ou a apresentar defesa, por escrito, no prazo de 30 (trinta) dias.
§ 1º - Decorrido o prazo previsto no “caput” deste artigo sem que haja o recolhimento ou acordo de parcelamento do débito fiscal ou a apresentação de defesa, o auto de infração será encaminhado à Delegacia Regional Tributária da circunscrição do autuado para a sua ratificação pelo Delegado Regional Tributário.
§ 2º - Após a ratificação do auto de infração, e encerrados os procedimentos de cobrança administrativa sem o devido recolhimento ou acordo de parcelamento, o débito fiscal será inscrito na dívida ativa.
§ 3º - Em caso de apresentação de defesa parcial, e não sendo recolhido ou parcelado o débito fiscal correspondente à exigência não impugnada, o órgão de julgamento providenciará a formação de processo em apartado para os fins previstos nos parágrafos anteriores, consignando-se essa circunstância mediante termo no processo original e prosseguindo-se no julgamento quanto às exigências impugnadas.
§ 4º - Considera-se parcial a defesa na qual o interessado não conteste, de forma expressa, um ou mais itens de acusação.

(...)

\textbf{5.	Queira o I. Perito Judicial demonstrar, a composição de cálculo dos valores apresentados no documento 1 (fls.31/34), e em comparação as fundamentações legais por “infringência” e “Capitulação da Multa”, responder se os mesmos cálculos correspondem corretamente com o apurado.:}

\textbf{Resposta:}	Conforme amplamente explanado no \hyperlink{3.1}{Subtópico 3.1 - Do valor exigido} do Laudo Pericial Contábil, a \hyperlink{tab1}{Tabela 1: Anexo ao AIIM n° 4.037.656-4 de 26/02/2014 (fl. 182 dos autos)} evidencia a evolução do cálculo impetrado no AIIM \textit{sub judice}, mais precisamente pela reprodução do Demonstrativo do Débito Fiscal, no qual se pode constatar que, o principal de imposto remonta em \textbf{R\$ 138.240,00} que, sendo base de cálculo para incidência da \textbf{taxa de juros nos patamares de 14,78\%} resulta em \textbf{R\$ 20.431,87} a título de juros, enquanto que, o valor básico para cálculo da multa punitiva remonta em \textbf{R\$ 844.800,00} que, no mesmo percentual de atualização aplicado, resulta em \textbf{R\$ 969.661,44}, figurando como base de cálculo para aplicação de 50\%, resultando em \textbf{R\$ 484.830,72} de multa punitiva imposta.
No que se refere a comparação com a capitulação legal, ressalva a \textbf{Perícia} que a “infringência” e “capitulação da multa”, não se configuram matéria a ser apreciada no objeto da prova pericial perquirida, mas sim, tão somente as taxas incididas sobre o principal (figurando os juros) e sobre o valor básico (figurando correção) para aplicação do percentual de multa de 50\%, do qual não se discute nesta fase processual.

\textbf{6.	Queira o I. Perito Judicial informar, se a coluna 9 do cálculo de fl. 33, demonstra a correta aplicação do §9º do art. 85 da lei 6.374/89?}

\textbf{Resposta:}	Considerando a referência sugerida no quesito, a coluna 9 do cálculo de fl. 33 dos autos é resultando da aplicação da taxa de 14,78\% (coluna 8) sobre o valor básico de R\$ 844.800,00 (coluna 6) resultando assim no importe de R\$ 969.661,44, (coluna 9), sendo que, assim tem-se a capitulação legal do §9º do art. 85 da Lei 6.374/89:

(...)

§ 9º - As multas previstas neste artigo, excetuadas as expressas em UFESP, devem ser calculadas sobre os respectivos valores básicos atualizados observando-se o disposto no artigo 96 desta lei; (NR)

(...)

Artigo \textbf{96} da Lei 6.374/89

(...)

Artigo 96 - O montante do imposto ou da multa, aplicada nos termos do artigo 85 desta lei, fica sujeito a juros de mora, que incidem: (NR)
I - relativamente ao imposto: (NR)
a) a partir do dia seguinte ao do vencimento, caso se trate de imposto declarado ou transcrito pelo fisco nos termos dos artigos 56 e 58 desta lei, de parcela devida por contribuinte enquadrado no regime de estimativa e de imposto exigido em auto de infração, nas hipóteses das alíneas “b”, “c”, “d”, “e”, “f”, “g”, “h”, “i”, “j” e “l” do inciso I do artigo 85 desta lei; (NR)
b) a partir do dia seguinte ao último do período abrangido pelo levantamento, caso se trate de imposto exigido em auto de infração na hipótese da alínea “a” do inciso I do artigo 85 desta lei; (NR)
c) a partir do mês em que, desconsiderada a importância creditada, o saldo tornar-se devedor, caso se trate de imposto exigido em auto de infração, nas hipóteses das alíneas “b”, “c”, “d”, “h”, “i” e “j” do inciso II do artigo 85 desta lei; (NR)
d) a partir do dia seguinte àquele em que ocorra a falta de pagamento, nas demais hipóteses; (NR)
II - relativamente à multa aplicada nos termos do artigo 85 desta lei, a partir do segundo mês subsequente ao da notificação da lavratura do auto de infração. (NR)
§ 1º - A taxa de juros de mora é equivalente: (NR) 
1 - por mês, à taxa referencial do Sistema Especial de Liquidação e de Custódia - SELIC para títulos federais, acumulada mensalmente; (NR)
2 - a 1\% (um por cento) para fração de mês, assim entendido qualquer período de tempo inferior a um mês; (NR)
§ 2º - Ocorrendo a extinção, substituição ou modificação da taxa prevista no item 1 do § 1º, o Poder Executivo adotará outro indicador oficial que reflita o custo do crédito no mercado financeiro. (NR)
§ 3º - O valor dos juros deve ser fixado e exigido na data do pagamento do débito fiscal, incluindo-se esse dia. (NR)
§ 4º - Na hipótese de auto de infração, pode o regulamento dispor que a fixação do valor dos juros se faça em mais de um momento. (NR)
§ 5º - A Secretaria da Fazenda divulgará, mensalmente, a taxa a que se refere este artigo. (NR)

(...)

Objetivamente, conclui-se que o percentual de 14,78\% adotado como taxa juros aplicados sobre o principal do imposto e adotada para correção do valor básico (base de cálculo) da multa punitiva é, fatidicamente, superior aos patamares da Taxa SELIC que, para o mesmo período de lavratura do AIIM, resulta em 10,87\% consoante parâmetros estabelecidos pela Receita Federal do Brasil - União.

\textbf{7.	Queira o I. Perito Judicial, demonstrar qual valor apurou em seus cálculos como montante principal, juros e multa punitiva à época e demonstrá-las:}

\textbf{Resposta:}	Conforme minuciosamente explanado no \hyperlink{3.3}{\emph{Subtópico 3.3 – Das Taxas de Juros e Correção Aplicadas em cotejo com a Taxa SELIC}} deste Laudo Pericial Contábil, a \textbf{Perícia} perseguiu a Taxa SELIC considerando os parâmetros de termo inicial (outubro de 2012) e data da lavratura do AIIM \textit{sub judice} (fevereiro de 2014), na própria fonte de cálculo dos débitos tributários administrados pela União (sítio da Receita Federal do Brasil), constatando que, a Taxa SELIC acumulada perfaz \textbf{10,87\%} na data da lavratura do AIIM.

Dessa forma, aplicando a taxa de \textbf{10,87\%} identificada no próprio programa que acumula a Taxa SELIC e que remunera os débitos federais (tese da \textbf{Requerente}), o resultado de juros perfaz o montante de \textbf{R\$ 15.026,69} sobre o principal, enquanto o valor básico da multa punitiva resulta no montante de \textbf{R\$ 936.629,76}, figurando como base de cálculo para aplicabilidade da multa calculada em \textbf{R\$ 468.314,88.}

Assim, apurou-se uma diferença total de \textbf{R\$ 21.921,02}
(somatório dos juros exigidos a maior da \textbf{Requerente} em
\textbf{R\$ 5.405,18} apurando-se, da mesma forma, a diferença a maior de \textbf{R\$ 16.515,84} de multa punitiva), em virtude da aplicação da taxa de 14,78\% em detrimento dos limites da Taxa SELIC Acumulada, a qual resultou em \textbf{10,87\%} para o mesmo período.